\documentclass{article}
\usepackage{listings}
\usepackage[utf8]{inputenc}

\title{ChainMDP description}
\author{Team 4}
\date{June 8th 2022}

\begin{document}

\maketitle

\section{Default Parameters}
\begin{flushleft}
These are default parameters for our agent. All values are provided in file chain\_test.py. There is no initial weight. 
\end{flushleft}
\begin{lstlisting}[language=Python]

sa_list = []


for i in range(env.n):
    for j in range(2):
        sa_list.append((i, j))

agent_params = {'gamma'            : 0.9,
                'kappa'            : 1.0,
                'mu0'              : 0.0,
                'lamda'            : 4.0,
                'alpha'            : 3.0,
                'beta'             : 3.0,
                'max_iter'         : 100,
                'sa_list'          : sa_list}
                
\end{lstlisting}

\section{Initialization}

Initialize agent by calling agent.
\begin{lstlisting}[language=Python]
agent = agent(agent_params).
\end{lstlisting}
\section{Traning method}
\begin{flushleft}
Below is the code used in training for k episodes. \ Just modify number in training(k) for training for k episodes. 
\end{flushleft}
\begin{lstlisting}[language=Python]

def training(k):

    for episode in range(k): 
        s = env.reset()
        done = False
      
        while not done:
            a = agent.take_action(s, 0)
          
            # Step environment
            s_, r, done, t = env.step(a)
            agent.observe([t, s, a, r, s_])
            agent.update_after_step(10, True)
          
            # Update current state
            s = s_
            
\end{lstlisting}



\bibliographystyle{unsrt}%Used BibTeX style is unsrt
\bibliography{ref}
\nocite{*}

\end{document}
